\section*{Chapter 3. Stochastic Processes: general theory}
\subsection*{3.1: Definition, distribution, and versions}
\begin{definition*}
Given $(\Omega,\F,\mathbf P)$, a \emph{stochastic process} (S.P.) $\{X_t\}$ is a collection of R.V.-s indexed by $t\in I$. We call $t \mapsto X_t(\omega)$ the \emph{sample path} of the S.P.
\end{definition*}

\begin{definition*}
A \emph{random walk} is the sequence $S_n = \sum_{i=1}^n \xi_i$, where $\xi_i$ are i.i.d. real-valued R.V.-s defined on the same $(\Omega,\F,\mathbf P)$. When $\xi_i \in \mathbb Z$ we say it's a random walk on the integers, and we call $\xi_i\in\{-1,1\}$ a simple random walk.
\end{definition*}

% do we care about this??
\begin{theorem*}
Consider the random walk $S_n$ when $\E\xi_i=0$ and $\E\xi_i^2=1$. Take the linear interpolation of $S_n$, scale space by $n^{-1/2}$ and time by $n^{-1}$. Taking $n\to\infty$ we get what we call \emph{Brownian motion} on $0\leq t \leq 1$.
\end{theorem*}

\begin{definition*} Given $N < \infty$ and $t_1,\dots,t_N\in \mathcal I$, we denote the (joint) distribution of $(X_{t_1},\dots,X_{t_N})$ by $F_{t_1,\dots,t_N}(\cdot)$, i.e. $F_{t_1,\dots,t_N}(\alpha_1,\dots,\alpha_N) = \mathbf P(X_{t_1 \leq \alpha_1},\dots,X_{t_N\leq\alpha_N})$ for all $\alpha_i\in\mathbb R$. We call the collection of functions $F_{t_1,\dots,t_N}(\cdot)$ the finite dimensional distributions (f.d.d.) of the S.P.
\end{definition*}

\begin{definition*}
With $\G_t$ the smallest $\sigma$-field containing $\sigma(X_s)$ for any $0\leq s\leq t$, we say that a S.P. $\{X_t\}$ has \emph{indepedent increments} if $X_{t+h}-X_t$ is independent of $\G_t$ for any $h>0$ and all $t \geq 0$. This property is determined by the f.d.d. That is, if $X_{t_1}, X_{t_2}-X_{t_1},\dots,X_{t_n} - X_{t_{n-1}}$ are mutually independent for all $n < \infty$ and $0 \leq t_1 < t_2 < \dots < t_n < \infty$ then the S.P. $\{X_t\}$ has independent increments.
\end{definition*}

\begin{remark*}
For example, both the random walk and the Brownian motion have indepedent increments.
\end{remark*}

\begin{example*}
Consider $\Omega = [0,1]$ with Borel $\sigma$-field and Uniform law. Define $Y_t(\omega) \equiv 0$ and $X_t(\omega) = \1_{\{\omega\}}(t)$. Note that $\mathbf P(X_t=Y_t)=1$ for any fixed $t$ but $\mathbf P(X_t=Y_t \, \forall t\in[0,1])=0$ and no sample paths $t \mapsto X_t(\omega)$ are continuous.
\end{example*}

\begin{definition*}
Two S.P. $\{X_t\}$ and $\{Y_t\}$ are called \emph{versions} of one another if they have the same f.d.d.s.
\end{definition*}

\begin{definition*}
A S.P. $\{Y_t\}$ is called a \emph{modification} of another S.P. $\{X_t\}$ if $\mathbf P(X_t = Y_t) = 1$ for all $t$.
\end{definition*}

\begin{exercise*}
If $\{Y_t\}$ is a modification of $\{X_t\}$ then $\{Y_t\}$ is also a version of $\{X_t\}$.
\end{exercise*}

\begin{example*}
Consider $\Omega = \{HH,TT,HT,TH\}$ with uniform probability measure, corresponding to two indepdent fair coin tosses with outcome $\omega=(\omega_1,\omega_2)$. Let $X_t(\omega) = \1_{[0,1)}(t)I_H(\omega_1) + \1_{[1,2)}(t)I_H(\omega_2)$ for $0\leq t<2$, and let $Y_t(\omega) = 1-X_t(\omega)$. These are version of each other but not modifications of each other.
\end{example*}

\begin{definition*}
We say that a collection of f.d.d.s is \emph{consistent} if 
\[
	\lim_{\alpha_k \uparrow \infty} F_{t_1,\dots,t_N}(\alpha_1,\dots,\alpha_N) = F_{t_1,\dots,t_{k-1},t_{k+1},\dots,t_N} (\alpha_1,\dots,\alpha_{k-1},\alpha_{k+1},\dots,\alpha_N),
\]
for any $1\leq k\leq N$, $t_1<t_2<\dots<t_N$ and $\alpha_i\in\mathbb R$.
\end{definition*}

\begin{proposition*}
The f.d.d.s of any S.P. must be consistent. Conversely, for any consistent collection f.d.d.s, there exists a probability space $(\Omega, \F,\mathbf P)$ and a stochastic process $\{X_t(\omega)\}$ on it, whose f.d.d.s are in agreement with the given collection. Further, the restriction of $\mathbf P$ to the $\sigma$-field $\F_X$ is uniquely determined by the given collection of f.d.d.
\end{proposition*}

%excluding the rest of 3.1 because don't have to know about cylindrical \sigma-field stuff

\subsection*{3.2: Characteristic functions, Gaussian variables and processes}
\begin{definition*}
A \emph{random vector} $\uX = (X_1,\dots,X_n)$ with values in $\mathbb R^n$ has the \emph{characteristic function} 
$\Phi_{\uX}(\utheta) = \E[\exp{(i \sum_{k=1}^n \theta_k X_k)}]$, 
where $\utheta = (\theta_1,\dots,\theta_n) \in \mathbb R^n$.
\end{definition*}

\begin{remark*}
The characteristic function exists for any $\uX$ because trig functions are bounded.
\end{remark*}

\begin{proposition*}
The characteristic function determines the law of a random vector. That is, if $\Phi_\uX(\utheta) = \Phi_{\underline{Y}}(\utheta)$ for all $\utheta$ then $\uX$ has the same law (= probability measure on $\mathbb R^n$) as $\underline Y$.
\end{proposition*}

\subsection*{3.3: Sample path continuity}

