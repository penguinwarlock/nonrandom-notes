\section*{Chapter 1. Probability, measure, and integration}
\subsection*{1.1. Probability spaces and $\sigma$-fields}
\begin{definition*}[$\sigma$-field]
We say that $\F \subset 2^\Omega$ is a $\sigma$-field (or a $\sigma$-algebra), if
\begin{enumerate}[label=(\alph*)]
\item $\Omega \in \F$,
\item If $A \in \F$, then $A^c \in \F$ as well,
\item If $A_i \in \F$ for $i = 1, 2, \dots$, then also $\cup_{i=1}^\infty A_i
	\in \F$.
\end{enumerate}
\end{definition*} 

\begin{definition*}[Probability Measure / Space]
A pair $(\Omega, \F)$ with $\F$ a $\sigma$-field of subsets of $\Omega$ is
called a \emph{measurable space}. Given a measurable space, a \emph{probability
measure} $\Pb$ is a function $\Pb: \F \rightarrow [0, 1]$, having the following
properties:
\begin{enumerate}[label=(\alph*)]
\item $0 \le \Pb(A) \le 1$ for all $A \in \F$,
\item $\Pb(\Omega) = 1$,
\item (Countable additivity) $\Pb(A) = \sum_{n=1}^\infty \Pb(A_n)$ whenever $A =
\cup_{n=1}^\infty A_n$ is a countable union of disjoint sets $A_n \in \F$.
\end{enumerate}
A \emph{probability space} is a triplet $(\Omega, \F, \Pb)$ with $\Pb$ a
probability measure on the measurable space $(\Omega, \F)$.
\end{definition*} 

\begin{definition*}[Generated $\sigma$-field]
Given a collection of subsets $A_\alpha \subseteq \Omega$, where $\alpha \in
\Gamma$ a not necessarily countable index set, we denote the smallest
$\sigma$-field $\F$ such that $A_\alpha \in \F$ for all $\alpha \in \Gamma$ by
$\sigma(\{A_\alpha\})$ (or sometimes by $\sigma(A_\alpha, \alpha \in \Gamma)$,
and call $\sigma(\{A_\alpha\})$ the $\sigma$-field \emph{generated} by the collection
$\{A_\alpha\}$. That is
\[
\sigma(\{A_\alpha\}) - \cap \{ \G : \G \subset 2^\Omega
\text{ is a $\sigma$-field}, A_\alpha \in \G \forall \alpha \in
\Gamma\}.
\] 
\end{definition*} 

\begin{lemma*} 
If two different collections of generators $\{A_\alpha\}$ and $\{B_\beta\}$ are
such that $A_\alpha \in \sigma(\{B_\beta\})$ for each $\alpha$ and $B_\beta \in
\sigma(\{A_\alpha\})$ for each $\beta$, then $\sigma(\{A_\alpha\}) =
\sigma(\{B_\beta\})$.
\end{lemma*} 

\begin{proposition*} 
There exists a subset of $\R$ that is not in $\B$. That is, not all sets are
Borel sets.
\end{proposition*} 


\subsection*{1.2: Random variables and their expectation}
\begin{definition*}[Random Variable]
A \emph{Random Variable} (R.V.) is a function $X: \Omega \rightarrow \R$ such
that $\forall \alpha \in \R$ the set $\{\omega: X(\omega) \le \alpha\}$ is in
$\F$ (such a function is also called a \emph{$\F$-measurable or, simply,
measurable function}).
\end{definition*} 

\begin{proposition*} 
For every R.V. $X(\omega)$ there exists a sequence of simple functions
$X_n(\omega)$ such that $X_n(\omega) \rightarrow X(\omega)$ as $n\rightarrow
\infty$, for each fixed $\omega \in \Omega$.
\end{proposition*} 

\begin{definition*}[Almost Surely]
We say that a R.V. $X$ and $Y$ are defined on the same probability space
$(\Omega, \F, \Pb)$ are almost surely the same if $\Pb(\{\omega: X(\omega) \ne
Y(\omega)\}) = 0$. This shall be denoted by $X \as Y$.
\end{definition*} 

\begin{definition*}
Given a R.V. $X$ we denote $\sigma(X)$ the smallest $\sigma$-field $\G \subseteq
\F$ such that $X(\omega)$ is measure on $(\Omega, \G)$.
\end{definition*} 

\begin{definition*} 
A function $g: \R \rightarrow \R$ is called \emph{Borel (measurable) function}
if $g$ is a R.V. on $(\R, \B)$. if $g$ is a R.V. on $(\R, \B)$.
\end{definition*} 





\subsection*{1.3: Converge of random variables}
\subsection*{1.4: Independence, weak convergence and uniform integrability}

